\documentclass[11pt, a4paper]{article}

% --- PREAMBUŁA (Uproszczona) ---
\usepackage[T1]{fontenc}
\usepackage[utf8]{inputenc}
\usepackage[polish]{babel} % Ustawienie języka polskiego
\usepackage[a4paper, top=2.5cm, bottom=2.5cm, left=2cm, right=2cm]{geometry}

% Pakiety matematyczne i graficzne
\usepackage{amsmath}
\usepackage{amssymb}
\usepackage{booktabs}
\usepackage{graphicx}
\usepackage{subcaption}
\usepackage{float}
\usepackage{hyperref}

\hypersetup{colorlinks=true, urlcolor=blue, linkcolor=black, citecolor=black}

\title{\vspace{-2cm}\textbf{Obliczenia Naukowe Laboratoria - Lista 4: Interpolacja Wielomianowa Newtona}}
\author{Kajetan Plewa}
\date{\today}

\begin{document}
	\maketitle
	\thispagestyle{empty}
	
	\section{Implementacja: Ilorazy Różnicowe (Zad. 1)}
	
	\subsection{Krótki opis problemu}
	Celem zadania było obliczenie współczynników $c_i$ wielomianu interpolacyjnego w postaci Newtona (ilorazów różnicowych) dla zadanych węzłów $x_i$ i wartości funkcji $f(x_i)$.  Wymogiem była implementacja algorytmu bez użycia tablicy dwuwymiarowej.
	
	\subsection{Rozwiązanie zaimplementowane w \texttt{ilorazyRoznicowe.jl}}
	Algorytm został zaimplementowany z użyciem rekurencji. \newline
	Dla każdego współczynnika $c_i$ korzystamy z przedstawionego na wykładzie wzoru: \newline
	\begin{equation}
		f[x_0, x_1, \dots, x_n] = \frac{f[x_1, x_2, \dots, x_n]-f[x_0, x_1, \dots, x_{n-1}]}{x_n-x_0}
	\end{equation}
	
	\newpage
	\section{Implementacja: Wartość Wielomianu w Postaci Newtona (Zad. 2)}
	
	\subsection{Krótki opis problemu}
	Zadanie polegało na obliczeniu wartości wielomianu $N_n(x)$ w zadanym punkcie $x=t$. Wymagano implementacji w czasie $O(n)$.
	
	\subsection{Rozwiązanie zaimplementowane w \texttt{newton.jl}}
	W celu osiągnięcia złożoności $O(n)$, skorzystano z postaci Newtona i \textbf{uogólnionego algorytmu Hornera}. Implementacja przetwarza współczynniki $c_i$ i węzły $x_i$ od najwyższego stopnia $n$ do $0$, iteracyjnie stosując wzór rekurencyjny
	\begin{equation}
		P_n(t) = c_i + (t - x_i) \cdot P_{n-1}(t)
	\end{equation}
	
	\section{Implementacja: Konwersja do Postaci Naturalnej (Zad. 3)}
	
	\subsection{Krótki opis problemu}
	Zadanie polegało na przekształceniu współczynników wielomianu z postaci Newtona ($c_i$) do postaci naturalnej ($a_i$), czyli $P_n(x) = a_n x^n + \dots + a_0$. Wymagana złożoność czasowa to $O(n^2)$.
	
	\subsection{Rozwiązanie zaimplementowane w \texttt{normalna.jl}}
	Zaimplementowany algorytm, podobnie jak poprzedni korzysta z własności postaci Newtona. Zastosowano podejście iteracyjne które \textit{rozwija} składniki postaci Newtona.
	\begin{equation}
		P_n(t) = c_i + (t - x_i) \cdot P_(n-1)(t)
	\end{equation}
	Poniżej znajduje się wytłumaczenie oznaczeń i działania kodu:
	\begin{itemize}
		\item Początkowo tablica $a$ kopiuje zawartość przekazanej tablicy z ilorazami różnicy.
		\item W każdej iteracji znajdujemy współczynniki postaci normalnych dla $P_(n-i)$ czyli dla wielomianu o stopniu $i$ - idziemy od tyłu czyli zaczynamy od:
		\begin{equation}
			Q_{n-1} = c_{n-1} + (x-x_{n-1})*a[n]
			\equiv Q_{n-1} = c_{n-1} + x*a[n] - x[n-1]*a[n]
		\end{equation} 
		\item Widać więc że wartość $a[n-1]$ spełnia równanie $a[n-1] = a[n-1] + x[n-1]*a[n]$. Będziemy z niego korzystać w uogólnionej formie:
		\begin{equation}
			a[i-1] = a[i-1] + x[i-1] * a[i]
		\end{equation}
		\item Dla $k$-tej iteracji ciąg $a[k] \dots a[n]$ będzie przechowywać poprawne współczynniki dla $Q_{n-k}$ w postaci normalnej.
	\end{itemize}
	Ostatecznie program zwraca $a$, które według powyższego załozenia przechowuje współczynniki opisujące postać normalną:
	\begin{equation}
		P(x) = a_nx^n + a_{n-1}x^{n-1} + \dots + a_0
	\end{equation}
	
	\newpage
	\section{Wybór Węzłów Interpolacyjnych (Zad. 4)}
	
	\subsection{Krótki opis problemu}
	Zadanie to wymagało implementacji funkcji wyboru węzłów do interpolacji na przedziale $[a, b]$, umożliwiając wybór między węzłami równoodległymi a węzłami Czebyszewa.
	
	\subsection{Rozwiązanie: Konfiguracja Węzłów}
	W pliku \texttt{ex4.jl} zaimplementowano funkcję \texttt{select\_nodes} obsługującą dwa typy węzłów, które są następnie wykorzystywane w głównej funkcji wizualizacyjnej \texttt{plotNewtonInterp}.
	
	\subsubsection{Węzły Równoodległe}
	Generowane są w równych odstępach $h = (b-a)/n$: $x_k = a + k \cdot h$.
	
	\subsubsection{Węzły Czebyszewa}
	Obliczenia oparto na pierwiastkach wielomianu Czebyszewa $T_{n+1}(x)$ - funkcja była podana na wykładzie nr 7:
	$$
	x_k = \frac{a+b}{2} + \frac{b-a}{2} \cdot \cos\left(\frac{2k+1}{2(n+1)}\pi\right)
	$$
	
	\newpage
	\section{Analiza Interpolacji dla Węzłów Równoodległych (Zad. 5)}
	
	Testy przeprowadzono dla funkcji $f(x) = e^x$ na $[0, 1]$ oraz $f(x) = x^2 \sin(x)$ na $[-1, 1]$, dla stopni $n=5, 10, 15$. 
	
	\subsection{Wyniki i ich interpretacja dla $f(x) = e^x$ na $[0, 1]$}
	% TUTAJ WPISZESZ ANALIZĘ I WYKRES
\begin{figure}[H]
	\centering
	\caption{Interpolacja $f(x) = |x|$ na $[-1, 1]$ (węzły równoodległe)}
	% Ustawienie szerokości na 0.3\textwidth, aby zmieścić 3 obrazy w jednym wierszu
	\subcaptionbox{$n=5$}{\centering \includegraphics[width=0.3\textwidth]{./graphs/fa_exp_n05_eq.png}} \quad
	\subcaptionbox{$n=10$}{\centering \includegraphics[width=0.3\textwidth]{./graphs/fa_exp_n10_eq.png}} \quad
	\subcaptionbox{$n=15$}{\centering 			\includegraphics[width=0.3\textwidth]{./graphs/fa_exp_n15_eq.png}}
	\caption{Interpolacja funkcji $f(x)=|x|$ z punktem nierozróżnialności $x=0$.}
	\label{fig:expX}
\end{figure}	
	\textbf{Interpretacja:} 
	
	\subsection{Wyniki i ich interpretacja dla $f(x) = x^2 \sin(x)$ na $[-1, 1]$}
	% TUTAJ WPISZESZ ANALIZĘ I WYKRES
	\begin{figure}[H]
		\centering
		\caption{Interpolacja $f(x) = |x|$ na $[-1, 1]$ (węzły równoodległe)}
		% Ustawienie szerokości na 0.3\textwidth, aby zmieścić 3 obrazy w jednym wierszu
		\subcaptionbox{$n=5$}{\centering \includegraphics[width=0.3\textwidth]{./graphs/fb_x2sinx_n05_eq.png}} \quad
		\subcaptionbox{$n=10$}{\centering \includegraphics[width=0.3\textwidth]{./graphs/fb_x2sinx_n10_eq.png}} \quad
		\subcaptionbox{$n=15$}{\centering 			\includegraphics[width=0.3\textwidth]{./graphs/fb_x2sinx_n15_eq.png}}
		\caption{Interpolacja funkcji $f(x)=x^2*\sin(x)$.}
		\label{fig:sinX}
	\end{figure}	
	
	\subsection{Interpretacja:} 
	
	\subsection{Wnioski:}
	
	\newpage
	\section{Analiza Zbieżności i Zjawisk Rozbieżności (Zad. 6)}
	
	Zadanie polegało na porównaniu węzłów równoodległych i Czebyszewa na przykładach prowadzących do niestabilności lub wolnej zbieżności.
	
	\subsection{Wyniki i ich interpretacja dla $f(x) = \frac{1}{1+x^2}$ na $[-5, 5]$ (Zjawisko Runge'go)}
	% TUTAJ WPISZESZ ANALIZĘ I WYKRESY
	
	\begin{figure}[H]
	\centering
	\caption{Interpolacja $f(x) = |x|$ na $[-1, 1]$ (węzły równoodległe)}
	% Ustawienie szerokości na 0.3\textwidth, aby zmieścić 3 obrazy w jednym wierszu
	\subcaptionbox{$n=5$}{\centering \includegraphics[width=0.3\textwidth]{./graphs/fb_runge_n05_eq.png}} \quad
	\subcaptionbox{$n=10$}{\centering \includegraphics[width=0.3\textwidth]{./graphs/fb_runge_n10_eq.png}} \quad
	\subcaptionbox{$n=15$}{\centering 			\includegraphics[width=0.3\textwidth]{./graphs/fb_runge_n15_eq.png}}
	\subcaptionbox{$n=5$}{\centering \includegraphics[width=0.3\textwidth]{./graphs/fb_runge_n05_czeb.png}} \quad
	\subcaptionbox{$n=10$}{\centering \includegraphics[width=0.3\textwidth]{./graphs/fb_runge_n10_czeb.png}} \quad
	\subcaptionbox{$n=15$}{\centering 			\includegraphics[width=0.3\textwidth]{./graphs/fb_runge_n15_czeb.png}}
	\caption{Interpolacja funkcji $f(x) = \frac{1}{1+x^2}$ .}
	\label{fig:aX}
\end{figure}	
	
	\textbf{Interpretacja (Węzły Równoodległe):} Dla tej funkcji, zwanej funkcją Runge'go, zwiększanie stopnia $n$ prowadzi do gwałtownego wzrostu oscylacji i błędu w pobliżu krańców przedziału. Jest to klasyczny przypadek **Zjawiska Runge'go**, gdzie interpolacja dla węzłów równoodległych jest niestabilna.
	
	\textbf{Interpretacja (Węzły Czebyszewa):} Zastosowanie węzłów Czebyszewa, charakteryzujących się zagęszczeniem przy brzegach, skutecznie eliminuje problem oscylacji. Błąd jest utrzymany na niskim poziomie w całym przedziale, potwierdzając optymalne własności tych węzłów.
	
	\subsection{Wyniki i ich interpretacja dla $f(x) = |x|$ na $[-1, 1]$}
	% TUTAJ WPISZESZ ANALIZĘ I WYKRESY
	

\begin{figure}[H]
	\centering
	\caption{Interpolacja $f(x) = |x|$ na $[-1, 1]$ (węzły równoodległe)}
	% Ustawienie szerokości na 0.3\textwidth, aby zmieścić 3 obrazy w jednym wierszu
	\subcaptionbox{$n=5$}{\centering \includegraphics[width=0.3\textwidth]{./graphs/fa_absX_n05_eq.png}} \quad
	\subcaptionbox{$n=10$}{\centering \includegraphics[width=0.3\textwidth]{./graphs/fa_absX_n10_eq.png}} \quad
	\subcaptionbox{$n=15$}{\centering 			\includegraphics[width=0.3\textwidth]{./graphs/fa_absX_n15_eq.png}}
	\subcaptionbox{$n=5$}{\centering \includegraphics[width=0.3\textwidth]{./graphs/fa_absX_n05_czeb.png}} \quad
	\subcaptionbox{$n=10$}{\centering \includegraphics[width=0.3\textwidth]{./graphs/fa_absX_n10_czeb.png}} \quad
	\subcaptionbox{$n=15$}{\centering 			\includegraphics[width=0.3\textwidth]{./graphs/fa_absX_n15_czeb.png}}
	\caption{Interpolacja funkcji $f(x) =  |x|$ .}
	\label{fig:abX}
\end{figure}	
	
	\textbf{Interpretacja:} Funkcja $f(x)=|x|$ nie jest różniczkowalna w punkcie $x=0$. Wielomian interpolacyjny próbuje wygładzić ten "narożnik". Mimo zwiększania $n$, błąd w otoczeniu $x=0$ pozostaje znaczący, co pokazuje, że brak gładkości funkcji limituje efektywność interpolacji wielomianowej, nawet przy zwiększaniu stopnia.
	
\subsection{Wnioski:}
	
\end{document}